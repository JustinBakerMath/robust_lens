\documentclass[11pt]{article}
\usepackage[utf8]{inputenc}
\usepackage{settings}

%%%%%%%%%
% TITLE %
%%%%%%%%%
\title{Robust Focusing of Heat Flux\\ Through a Thermal Lens}
\date{}

%%%%%%%%%%%%
% DOCUMENT %
%%%%%%%%%%%%
\begin{document}
\maketitle

%%%%%%%%%%%%
% ABSTRACT %
%%%%%%%%%%%%
\begin{abstract}
    
\end{abstract}

%%%%%%%%%
% INTRO %
%%%%%%%%%
\section{Introduction}

\subsection{Notation}

%%%%%%%%
% BODY %
%%%%%%%%
\section{Robust Thermal Lens}

A thermal lens is the optimal distribution of non-homogeneous materials $\chi_1$ and $\chi_2$ such that the energy functional

\begin{align*}
	J = \int_{\Gamma_3}\rho(\Gamma)q\cdot n(\Gamma)d\Gamma
\end{align*}

is maximized.


The thermal lens is made robust by subjecting the constraint that the energies imposed by the fluxes $q_1$ and $q_2$ are equivalent.

\begin{align*}
	J &= \int_{\Gamma_3}\rho(\Gamma)q\cdot n(\Gamma)d\Gamma \\
	\text{s.t. } & \int_{\Gamma_3}\rho(\Gamma)q_1\cdot n(\Gamma)d\Gamma = \int_{\Gamma_3}\rho(\Gamma)q_2\cdot n(\Gamma)d\Gamma 
\end{align*}


%%%%%%%%%%%
% RESULTS %
%%%%%%%%%%%
\section{Numerical Results}

%%%%%%%%%%%%%%%%
% BIBLIOGRAPHY %
%%%%%%%%%%%%%%%%
\newpage
\bibliographystyle{abbrv}
\bibliography{references.bib}

%%%%%%%%%%%%
% APPENDIX %
%%%%%%%%%%%%
\newpage
\begin{appendix}
\section{Algorithms}
\begin{align*}
\lambda_+ &= \alpha\theta+\beta(1-\theta)\\[2ex]
\lambda_- &= \bigg(\frac{\theta}{\alpha} + \frac{1-\theta}{\beta}\bigg)^{-1}
\end{align*}

\begin{align*}
A^* &= \begin{pmatrix} \cos(\phi) & \sin(\phi)\\ -\sin(\phi) & \cos(\phi)\end{pmatrix}
\begin{pmatrix} \lambda_+ & 0\\ 0 & \lambda_-\end{pmatrix}
\begin{pmatrix} \cos(\phi) & -\sin(\phi)\\ \sin(\phi) & \cos(\phi)\end{pmatrix}\\[2ex]
&=\begin{pmatrix}
\lambda_+\cos^2(\phi)+\lambda_-\sin^2(\phi) & (-\lambda_++\lambda_-)\sin(\phi)\cos(\phi)
\\ (-\lambda_++\lambda_-)\sin(\phi)\cos(\phi) & \lambda_+\sin^2(\phi)+\lambda_-\cos^2(\phi)
\end{pmatrix}\end{align*}

\begin{align*}
\frac{\partial\lambda_-}{\partial\theta} &= -\bigg(\frac{1}{\alpha} - \frac{1}{\beta}\bigg)\bigg(\frac{\theta}{\alpha} + \frac{1-\theta}{\beta}\bigg)^{-2}\\
\frac{\partial\lambda_+}{\partial\theta} &= \alpha-\beta\\
\end{align*}

\begin{align*}
\frac{\partial A}{\partial\theta} &= \begin{pmatrix} \cos(\phi) & \sin(\phi)\\ -\sin(\phi) & \cos(\phi)\end{pmatrix}
\begin{pmatrix} \frac{\partial\lambda_+}{\partial\theta} & 0\\ 0 & \frac{\partial\lambda_-}{\partial\theta}\end{pmatrix}
\begin{pmatrix} \cos(\phi) & -\sin(\phi)\\ \sin(\phi) & \cos(\phi)\end{pmatrix}\\
\end{align*}

\begin{align*}
\frac{\partial A}{\partial\phi} &= \begin{pmatrix} (-\lambda_++\lambda_-)(2\cos(\phi)\sin(\phi)) & (-\lambda_++\lambda_-)(\cos^2(\phi)-\sin^2(\phi))
\\ (-\lambda_++\lambda_-)(\cos^2(\phi)-\sin^2(\phi)) & (\lambda_+-\lambda_-)(2\cos(\phi)\sin(\phi)) \end{pmatrix}\end{align*}

* Solve $u$

\begin{align*}
-\nabla \cdot (A^* \nabla u) &= 0\\
\frac{d}{dn}(A^*\nabla u)\vert_{\Gamma_1} &= 1\\
\frac{d}{dn}(A^*\nabla u)\vert_{\Gamma_2} &= 0\\
u\vert_{\Gamma_3}&=0
\end{align*}

* Solve $p$

\begin{align*}
\nabla \cdot (A^* \nabla p) &= 0\\
\frac{d}{dn}(A^*\nabla p)\vert_{\Gamma_1} &= 0\\
\frac{d}{dn}(A^*\nabla p)\vert_{\Gamma_2} &= 0\\
p\vert_{\Gamma_3}&=\rho
\end{align*}


* Update $\theta$
\begin{align*}
\theta_{k+1} = \max \bigg( 0,\min \bigg(1,\theta_k-t_k\bigg(\ell_k+\frac{\partial A^*}{\partial \theta}(\theta_k,\phi_k)\nabla u_k\cdot \nabla p_k\bigg) \bigg)\bigg)\end{align*}

* Update $\phi$
\begin{align*}
\phi_{k+1} = \phi_k-t_k\frac{\partial A^*}{\partial \theta}(\theta_k,\phi_k)\nabla u_k\cdot \nabla p_k\end{align*}

**Extended Functional**

\begin{align*}
J = \int_{\Gamma_3}\rho(\Gamma)q\cdot n(\Gamma)d\Gamma + \iint_S \mu \nabla\cdot (D_0\nabla T)ds
\end{align*}

**Primal**

\begin{align*}
-\nabla \cdot (A^* \nabla u) &= 0\\
\frac{d}{dn}(A^*\nabla u)\vert_{\Gamma_1} &= 1\\
\frac{d}{dn}(A^*\nabla u)\vert_{\Gamma_2} &= 0\\
u\vert_{\Gamma_3}&=0
\end{align*}

**Dual**

\begin{align*}
\nabla \cdot (A^* \nabla p) &= 0\\
\frac{d}{dn}(A^*\nabla p)\vert_{\Gamma_1} &= 0\\
\frac{d}{dn}(A^*\nabla p)\vert_{\Gamma_2} &= 0\\
p\vert_{\Gamma_3}&=\rho
\end{align*}

\end{appendix}

\end{document}
